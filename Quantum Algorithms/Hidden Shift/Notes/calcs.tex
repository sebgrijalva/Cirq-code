
%\documentclass[aps,pra,amsmath,amssymb,amsfonts,superscriptaddress,floatfix]{revtex4} 
\documentclass[letterpaper,onecolumn]{quantumarticle}
\pdfoutput=1
\usepackage{amsmath,amsthm,amsfonts,amssymb}
\usepackage{graphicx}

\usepackage{color}
\usepackage{hyperref}
%\numberwithin{equation}{section}
%\numberwithin{figure}{section}

\newtheorem{theorem}{Theorem}
\newtheorem{corollary}{Corollary}
\newtheorem{lemma}{Lemma}
\newtheorem{remark}{Remark}
\newtheorem{note}[theorem]{Note}
\newtheorem{definition}{Definition}
\newtheorem{assumption}{Assumption}
\usepackage{algorithm}
\usepackage[capitalize,nameinlink]{cleveref}

\newcommand{\ocite}{~\cite}


\newcommand{\tr}{{\rm tr}}
\newcommand{\be}{\begin{equation}}
\newcommand{\ee}{\end{equation}}
\newcommand{\Sw}{{\rm Swap}}
\newcommand{\Id}{{\rm Id}}
\newcommand{\cO}{{\cal O}}
\newcommand{\cC}{{\cal C}}
\newcommand{\cD}{{\cal D}}
\newcommand{\cE}{{\cal E}}
\newcommand{\expec}{\mathbb{E}}
\newcommand{\poly}{{\rm poly}}
\newcommand{\qpoly}{{\rm qpoly}}
\newcommand{\epe}{E_{PE}}
\newcommand{\Pbias}{P_{coin}}
\newcommand{\Ocl}{{\rm AdNSP}}
\newcommand{\cH}{{\cal H}}
\newcommand{\isL}{{\cal L}}
\newcommand{\glen}{L}
\begin{document}

\title{Notes on the implementation of the Hidden Shift Algorithm}

\author{Sebastián Grijalva}
\affiliation{github.com/sebgrijalva}

\begin{abstract}
These are some calculations that give more detail about how the implementation of the Hidden Shift Algorithm works.
\end{abstract}
\maketitle

\section{Introduction}

Hidden Shift Algorithms are interesting, the example given is written in \textsf{Cirq}.


\subsection{Problem Statement and Results}

Consider the set of $N$-bit strings $\{0,1\}^N$. Let $f$ and $g$ be two function oracles $f,g : \{0,1\}^N \to \{0,1\}^N$, which are the same up to a hidden bit string $s \in \{0,1\}^N$ such that $g(x) = f(x \oplus s)$. The Hidden Shift Algorithm determines s by quering the two oracles. The 
implementation in the example considers the following definition (called \emph{bent} function):

\be 
\label{oracle}
f(x) = \sum_i x_{2i-1} x_{2i}
\ee

where $x_i$ is the $i$-th bit of $x$. While a classical algorithm requires $\sim 2^{N/2}$ queries, the Hidden Shift 
Algorithm solves the problem in $O(1)$ steps. We thus have an \emph{exponential}
reduction.


\section{Application of Quantum Gates to the initial state}
\label{gates}

\subsection{Description of the Algorithm}

Let us call the initial state $|0^N\rangle$ by which we will mean the tensor product $|0\rangle \otimes \cdots \otimes |0\rangle$, $N$ times. After application of the first Hadamard set of gates, we create the superposition 

$$H^N |0^N \rangle = \frac{1}{\sqrt{2}^N} \sum_{x\in \{0,1\}^N} |x\rangle$$

\noindent Now we apply the shift $s$ by means of a set of $\textsf{X}$ gates representing the bit-string $s$ at each position of the circuit (the flip induced by the $\textsf{X}$ gate is equivalent to the modulo-$2$ addition that represents the action of the shift). By straightforwardly using  the linearity of this operation, we obtain the state 

$$\frac{1}{\sqrt{2}^N} \sum_{x\in {0,1}^N} |x \oplus s\rangle$$

\noindent At this point we apply the oracle function (\ref{oracle}) by means of a series of \textsf{Controlled-Z} gates: To see this, consider first the action of $\textsf{CZ}$ on two succesive sites $a,a+1$:

$$
\textsf{CZ}(a,a+1) |\cdots x_a x_{a+1} \cdots \rangle= (-1)^{x_a x_{a+1}} |\cdots x_a x_{a+1} \cdots \rangle
$$

\noindent indeed, if $x_a$ is ``active'' (i.e. 1), the phase of the state becomes the eigenvalue of $| \cdots x_{a+1} \cdots \rangle$. Applyting this to each pair of channels, we reproduce the function \emph{on the phases} of the amplitudes of each state $|x \oplus s \rangle$:

$$\frac{1}{\sqrt{2}^N} \sum_{x\in \{0,1\}^N} (-1)^{f(x\oplus s)}|x \oplus s\rangle$$

\noindent Notice that this sum can be equivalently translated over the shift $s$. More explicitly, for any function $\kappa$:

$$\sum_{x\in \{0,1\}^N} \kappa(x\oplus s) |x \oplus s\rangle  =\sum_{\underset{x\in \{0,1\}^N}{x\oplus s}} \kappa(x\oplus s \oplus s) |x \oplus s \oplus s\rangle= \sum_{x\in \{0,1\}^N} \kappa(x)|x \rangle$$

This means that our state can be written simply as 

$$\frac{1}{\sqrt{2}^N} \sum_{x\in \{0,1\}^N} (-1)^{f(x)}|x \rangle$$

 We apply a shift $s$ set of gates again (recall that this is not an operation on the phases) to arrive at $\frac{1}{\sqrt{2}^N} \sum_{x\in {0,1}^N} (-1)^{f(x)}|x \oplus s\rangle$. At this point we apply another full set of Hadamard gates:
 
$$\frac{1}{\sqrt{2}^{2N}} \sum_{x\in {0,1}^N} (-1)^{f(x)} \sum_{y \in \{ 0,1 \}^N} (-1)^{x\oplus s\cdot y}|y \rangle $$
 
rearranging the sum we get:

$$\frac{1}{\sqrt{2}^{2N}} \sum_{x,y\in {0,1}^N}   (-1)^{f(x )+x\cdot y}(-1)^{y\cdot s}|y \rangle $$

we make now the following definition:

\be
\label{fourier}
\hat{F}(y) \equiv \frac{1}{\sqrt{2}^N} \sum_{x\in\{0,1\}^N} (-1)^{f(x)+x\cdot y}
\ee

Applying this to our state, we obtain:

$$\frac{1}{\sqrt{2}^{2N}} \sum_{x,y\in {0,1}^N}   (-1)^{f(x )+x\cdot y}(-1)^{y\cdot s}|y \rangle = \frac{1}{\sqrt{2}^{N}} \sum_{x,y\in {0,1}^N}   \hat{F}(y)(-1)^{y\cdot s}|y \rangle$$


Then we have the following result:

\begin{lemma}
\label{fourier_identity}
Let $F$ be defined with (\ref{oracle}) as $F(y) = (-1)^{f(y)}$, and $\hat{F}$ as in (\ref{fourier}). Then:

\be
\label{identity}
F(y)\hat{F}(y)  = 1
\ee

\begin{proof}
Notice that $N$ should be even to make $f$ well defined. First consider the case where $N=2$:
%
\begin{eqnarray*}
F(y)\hat{F}(y)  &= \frac{1}{\sqrt{2}^2} \displaystyle\sum_{x\in \{00,01,10,11 \}} (-1)^{y_1y_2}(-1)^{x_1x_2}(-1)^{x_1y_1+x_2y_2}\\
&=\frac{(-1)^{y_1y_2}}{2} \Big(1+(-1)^{y_1}+(-1)^{y_2} - (-1)^{y_1+y_2} \Big)\\
\end{eqnarray*}
%
we see from here that for $y\in\{00,01,10,11\}$, the right hand side gives 1. Now, the idea is to decompose the general sum two qubits at a time:
%
\begin{eqnarray*}
F(y)\hat{F}(y)  &= \frac{1}{\sqrt{2}^{N}} \displaystyle\sum_{x\in \{0,1 \}^{N}} (-1)^{\sum_i y_{2i-1}y_{2i}}(-1)^{\sum_i x_{2i-1}x_{2i}}(-1)^{\sum_i x_iy_i}\\
&= \frac{1}{\sqrt{2}^{N}} \displaystyle\sum_{x\in \{0,1 \}^{N}} \prod_i (-1)^{ y_{2i-1}y_{2i}}(-1)^{ x_{2i-1}x_{2i}}(-1)^{ x_iy_i}\\
&=  \displaystyle\prod_{i=1}^N  \left\{\frac{1}{\sqrt{2}^{2}}\sum_{x_i\in \{0,1 \}^{2}}  (-1)^{ y_{2i-1}y_{2i}}(-1)^{ x_{2i-1}x_{2i}}(-1)^{ x_iy_i} \right\}\\
\end{eqnarray*}
%
The last line is a product of terms that we have seen give each 1.

\end{proof}
\end{lemma}

By this lemma we see that an application of the function $f$ via \textsf{Controlled-Z} gates transforms the state into:

$$\frac{1}{\sqrt{2}^{N}} \sum_{x,y\in {0,1}^N}   F(y)\hat{F}(y)(-1)^{y\cdot s}|y \rangle = \frac{1}{\sqrt{2}^{N}} \sum_{y\in {0,1}^N} (-1)^{y\cdot s}|y \rangle = H^N |s\rangle$$

Finally, we apply once more a full set of Hadamards to return to the desired $s$ shift state (since the inverse of a Hadamard is another Hadamard).


\bibliographystyle{unsrturl}
\bibliography{calcs}
\end{document}
